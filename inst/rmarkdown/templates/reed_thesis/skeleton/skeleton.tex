% This is the Reed College LaTeX thesis template. Most of the work
% template. Later comments etc. by Ben Salzberg (BTS). Additional
% restructuring and APA support by Jess Youngberg (JY).
% Your comments and suggestions are more than welcome; please email
% them to cus@reed.edu
%
% See http://web.reed.edu/cis/help/latex.html for help. There are a
% great bunch of help pages there, with notes on
% getting started, bibtex, etc. Go there and read it if you're not
% already familiar with LaTeX.
%
% Any line that starts with a percent symbol is a comment.
% They won't show up in the document, and are useful for notes
% to yourself and explaining commands.
% Commenting also removes a line from the document;
% very handy for troubleshooting problems. -BTS

% As far as I know, this follows the requirements laid out in
% the 2002-2003 Senior Handbook. Ask a librarian to check the
% document before binding. -SN

%%
%% Preamble
%%
% \documentclass{<something>} must begin each LaTeX document
\documentclass[12pt,twoside]{reedthesis}
% Packages are extensions to the basic LaTeX functions. Whatever you
% want to typeset, there is probably a package out there for it.
% Chemistry (chemtex), screenplays, you name it.
% Check out CTAN to see: http://www.ctan.org/
%%
\usepackage{graphicx,latexsym}
\usepackage{amssymb,amsthm,amsmath}
\usepackage{longtable,booktabs,setspace}
\usepackage{chemarr} %% Useful for one reaction arrow, useless if you're not a chem major

% Modified by CII
\usepackage[hyphens]{url}
\usepackage{hyperref}

\usepackage{rotating}

% Added by CII
%\usepackage{hanging}
%\usepackage{natbib}

\usepackage[backend=bibtex]{biblatex}
\addbibresource{thesis}
% \bibliographystyle{APA/apa-good}  % or
% \bibliography{thesis}


% Comment out the natbib line above and uncomment the following two lines to use the new
% biblatex-chicago style, for Chicago A. Also make some changes at the end where the
% bibliography is included.
%\usepackage{biblatex-chicago}
%\bibliography{thesis}

% \usepackage{times} % other fonts are available like times, bookman, charter, palatino

\title{My Final College Paper}
%\author{Your R. Name}
\author{Your R. Name}
% The month and year that you submit your FINAL draft TO THE LIBRARY (May or December)
\date{May 20xx}
\division{Mathematics and Natural Sciences}
\advisor{Advisor F. Name}
%If you have two advisors for some reason, you can use the following
%\altadvisor{Your Other Advisor}
%%% Remember to use the correct department!
\department{Mathematics}
% if you're writing a thesis in an interdisciplinary major,
% uncomment the line below and change the text as appropriate.
% check the Senior Handbook if unsure.
%\thedivisionof{The Established Interdisciplinary Committee for}
% if you want the approval page to say "Approved for the Committee",
% uncomment the next line
%\approvedforthe{Committee}

\setlength{\parskip}{0pt}
%%
%% End Preamble
%%
%

% Below added by CII

\renewcommand{\contentsname}{Table of Contents}
\renewcommand{\bibname}{References}

\providecommand{\tightlist}{%
  \setlength{\itemsep}{0pt}\setlength{\parskip}{0pt}}

\Acknowledgements{
I want to thank a few people.
}

\Dedication{
You can have a dedication here if you wish.
}

\Preface{
This is an example of a thesis setup to use the reed thesis document
class.
}

\Abstract{
The preface pretty much says it all.
}

\begin{document}

      \maketitle
  
  \frontmatter % this stuff will be roman-numbered
  \pagestyle{empty} % this removes page numbers from the frontmatter

      \begin{acknowledgements}
      I want to thank a few people.
    \end{acknowledgements}
  
      \begin{preface}
      This is an example of a thesis setup to use the reed thesis document
      class.
    \end{preface}
  
  % Add table of abbreviations?

    {
    \hypersetup{linkcolor=black}
    \setcounter{tocdepth}{2}
    \tableofcontents
  }
  
      \listoftables
  
      \listoffigures
  
      \begin{abstract}
      The preface pretty much says it all.
    \end{abstract}
  
      \begin{dedication}
      The preface pretty much says it all.
    \end{dedication}
  
%  
  \mainmatter % here the regular arabic numbering starts
  \pagestyle{fancyplain} % turns page numbering back on

  \chapter*{Introduction}
  
  \addcontentsline{toc}{chapter}{Introduction}
  
  \chaptermark{Introduction} \markboth{Introduction}{Introduction}
  
  Welcome to the \LaTeX~thesis template. If you've never used \TeX~or
  \LaTeX~before, you'll have an initial learning period to go through, but
  the results of a nicely formatted thesis are worth it for more than the
  aesthetic benefit: markup like \LaTeX~is more consistent than the output
  of a word processor, much less prone to corruption or crashing and the
  resulting file is smaller than a Word file. While you may have never had
  problems using Word in the past, your thesis is going to be about twice
  as large and complex as anything you've written before, taxing Word's
  capabilities. If you're still on the fence about using \LaTeX, read the
  Introduction to LaTeX on the CUS site as well as skim the following
  template and give it a few weeks. Pretty soon all the markup gibberish
  will become second nature.
  
  \section{Why use it?}
  
  \LaTeX~does a great job of formatting tables and paragraphs. Its
  line-breaking algorithm was the subject of a PhD.\thinspace thesis. It
  does a fine job of automatically inserting ligatures, and to top it all
  off it is the only way to typeset good-looking mathematics.
  
  \section{Who should use it?}
  
  Anyone who needs to use math, tables, a lot of figures, complex
  cross-references, IPA or who just cares about the final appearance of
  their document should use \LaTeX. At Reed, math majors are required to
  use it, most physics majors will want to use it, and many other science
  majors may want it also.
  
  \chapter{R Markdown}
  
  This is an R Markdown document. Markdown is a simple formatting syntax
  for authoring HTML, PDF, and MS Word documents. For more details on
  using R Markdown see \url{http://rmarkdown.rstudio.com}.
  
  When you click the \textbf{Knit} button a document will be generated
  that includes both content as well as the output of any embedded R code
  chunks within the document. You can embed an R code chunk like this:
  
  \begin{CodeChunk}
  \begin{CodeInput}
  summary(cars)
  \end{CodeInput}
  \begin{CodeOutput}
       speed           dist       
   Min.   : 4.0   Min.   :  2.00  
   1st Qu.:12.0   1st Qu.: 26.00  
   Median :15.0   Median : 36.00  
   Mean   :15.4   Mean   : 42.98  
   3rd Qu.:19.0   3rd Qu.: 56.00  
   Max.   :25.0   Max.   :120.00  
  \end{CodeOutput}
  \end{CodeChunk}
  
  Testing citations: \cite{angel2000} {[}@angel2000{]}
  
  You can also embed plots, for example:
  
  \begin{CodeChunk}
  \begin{CodeInput}
  plot(cars)
  \end{CodeInput}
  
  
  \begin{center}\includegraphics{skeleton_files/figure-latex/plot-1} \end{center}
  
  \end{CodeChunk}
  
  \begin{CodeChunk}
  \begin{CodeInput}
  plot(Sepal.Length ~ Species, data = iris)
  \end{CodeInput}
  
  
  \begin{center}\includegraphics{skeleton_files/figure-latex/unnamed-chunk-1-1} \end{center}
  
  \end{CodeChunk}
  
  \section{Bibliographies}
  
  Of course you will need to cite things, and you will probably accumulate
  an armful of sources. This is why BibTeX was created. For more
  information about BibTeX and bibliographies, see our CUS site
  (\url{http://web.reed.edu/cis/help/latex/index.html})\cite{reedweb2007}
  {[}@reedweb2007{]}. There are three pages on this topic: \emph{bibtex}
  (which talks about using BibTeX, at
  \url{http://web.reed.edu/cis/help/latex/bibtex.html}),
  \emph{bibtexstyles} (about how to find and use the bibliography style
  that best suits your needs, at
  \url{http://web.reed.edu/cis/help/latex/bibtexstyles.html}) and
  \emph{bibman} (which covers how to make and maintain a bibliography by
  hand, without BibTeX, at
  \url{http://web.reed.edu/cis/help/latex/bibman.html}). The last page
  will not be useful unless you have only a few sources. There used to be
  APA stuff here, but we don't need it since I've fixed this with my
  apa-good natbib style file.

\printbibliography

    % Index?

\end{document}

